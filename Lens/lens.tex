\documentclass[tikz]{standalone}

% A representation of the lens space L(p, q) fundamental domain inspired by figure in
% Hatcher, Algebraic topology, p 145
% https://pi.math.cornell.edu/~hatcher/AT/AT.pdf

% Despite using the tikz option, the cropping is not perfect.
% I find that there is a bit more whitespace on the left of the image than on the right.
% Externally cropping fixes this ...?

% Version 1.0.

\usepackage{tikz-3dplot}
% Non-standard package: download from
% https://github.com/matthias-wolff/tikz-3dplot-circleofsphere
\usepackage{tikz-3dplot-circleofsphere}
\usepackage{xcolor}

\begin{document}

% Lens space L(p,q)
% No error checking is performed!
% p must be no larger than the number of colors defined below.
% As p gets larger the cap gets flatter and flatter. p > 7 becomes hard to read.
% It is recommended to stick to 2 <= p <= 7.
\def\lensp{7}
\def\lensq{2}

% Cap parameters: radius and viewing angle with respect to the z-axis.
\pgfmathsetmacro{\lenssphereradius}{10}
\pgfmathsetmacro{\lensviewtheta}{7} % degrees

% Additional rotation angle around the z-axis.
% Tweak for aesthetic reasons to align the wedges "nicely".
\pgfmathsetmacro{\lensphi}{-5} % degrees (additional rotation angle around z-axis, tweak for aesthetics)
% Half angular size of the cap.
% This makes the caps quite flat for large p, but correctly produces the requested lens space.
% This can be set to an arbitrary value for any L(p,q) at the expense of not accurately representing the space.
\pgfmathsetmacro{\lenscapangle}{180/\lensp} % degrees


% A fudge factor to draw the top and bottom of the lens.
% Due to the rotated view this is not just 1, i.e., we do not want the top and bottom
% of the lens to start/end at \lenscapangle.
% Here is an approximation that seems to work. Maybe the equator line is thick enough to hide small imperfections.
% Obviously this cannot be correct when the angle goes to 90 deg!
\pgfmathsetmacro{\lensviewfudgefactor}{1 / cos(\lensviewtheta)}

\pgfmathsetmacro{\axislen}{1.5} % length of rotation axis to draw
% Extra rotation in southern cap for L(p, q): Rotation is by 2 pi q / p
\pgfmathsetmacro{\southlensphi}{\lensphi - 360*\lensq / \lensp}

% Color cycler for wedges. There must be at least \lensp of these.
% Here we specify the category 10 colors
\definecolor{C1}{HTML}{1f77b4}
\definecolor{C2}{HTML}{ff7f0e}
\definecolor{C3}{HTML}{2ca02c}
\definecolor{C4}{HTML}{d62728}
\definecolor{C5}{HTML}{9467bd}
\definecolor{C6}{HTML}{8c564b}
\definecolor{C7}{HTML}{e377c2}
\definecolor{C8}{HTML}{7f7f7f}
\definecolor{C9}{HTML}{bcbd22}
\definecolor{C10}{HTML}{17becf}

% The base coordinate system will have (0, 0, 0) at the center of the "equator" between the two caps
\pgfmathsetmacro{\zoriginshift}{\lenssphereradius * cos(\lenscapangle)}

% Set our main coordinate system view.
\tdplotsetmaincoords{90-\lensviewtheta}{0}

% For clipping arcs in front of and behind lens.
% See https://tex.stackexchange.com/questions/539956/how-can-i-draw-sphere-cut-by-plane-by-using-tikz-3dplot
\makeatletter
\tikzset{
    reuse path/.code={\pgfsyssoftpath@setcurrentpath{#1}}
}
\makeatother

\begin{tikzpicture}[tdplot_main_coords, >=stealth, thick, scale=0.75]
  \coordinate (NorthSphereOrigin) at (0, 0, -\zoriginshift);
  \coordinate (SouthSphereOrigin) at (0, 0, \zoriginshift);
 
  % Top Half arcs
  \tdplotsetrotatedcoords{0}{0}{0}
  \tdplotsetrotatedcoordsorigin{(NorthSphereOrigin)}
  \foreach \j in {1,...,\lensp} {
    \pgfmathsetmacro{\arcang}{360*\j/\lensp+\lensphi}
    % Make paths to allow styling the front and back of arcs using clipping
    \begin{scope}[shift=(NorthSphereOrigin), overlay]
      \tdplotCsDrawLonCircle[tdplotCsFront/.style={draw=none, save path=\pathFG},tdplotCsBack/.style={draw=none, save path=\pathBG}]{\lenssphereradius}{-90+\arcang}
    \end{scope}
    \tdplotsetrotatedthetaplanecoords{\arcang}
    % Draw foreground arcs
    \begin{scope}
      \clip [reuse path=\pathFG];
      \tdplotdrawarc[tdplot_rotated_coords, very thick, color=C\j]{(NorthSphereOrigin)}{\lenssphereradius}{0}{\lenscapangle}{}{}
    \end{scope}
    % Draw background arcs
    \begin{scope}
      \clip [reuse path=\pathBG];
      \tdplotdrawarc[tdplot_rotated_coords, very thick, color=C\j, dashed]{(NorthSphereOrigin)}{\lenssphereradius}{0}{\lenscapangle}{}{}
    \end{scope}
  }

  % Bottom Half arcs
  \tdplotsetrotatedcoords{0}{0}{0}
  \tdplotsetrotatedcoordsorigin{(SouthSphereOrigin)}
  \foreach \j in {1,...,\lensp} {
    \pgfmathsetmacro{\arcang}{360*\j/\lensp+\southlensphi}
    % Make paths to allow styling the front and back of arcs using clipping
    \begin{scope}[shift=(SouthSphereOrigin), overlay]
      \tdplotCsDrawLonCircle[tdplotCsFront/.style={draw=none, save path=\pathFG},tdplotCsBack/.style={draw=none, save path=\pathBG}]{\lenssphereradius}{-90+\arcang}
    \end{scope}
    \tdplotsetrotatedthetaplanecoords{180+\arcang}
    % Draw foreground arcs
    \begin{scope}
      \clip [reuse path=\pathFG];
      \tdplotdrawarc[tdplot_rotated_coords, very thick, color=C\j]{(SouthSphereOrigin)}{\lenssphereradius}{-180}{-180+\lenscapangle}{}{}
    \end{scope}
    % Draw background arcs
    \begin{scope}
      \clip [reuse path=\pathBG];
      \tdplotdrawarc[tdplot_rotated_coords, very thick, color=C\j, dashed]{(SouthSphereOrigin)}{\lenssphereradius}{-180}{-180+\lenscapangle}{}{}
    \end{scope}
  }

  % Angle where the cap intersects the "equator"
  \pgfmathsetmacro{\capendangle}{\lensviewfudgefactor*(90-\lenscapangle)}

  % Top cap
  \begin{scope}[shift=(NorthSphereOrigin)]
    \draw[tdplot_screen_coords, black] (\capendangle:\lenssphereradius) arc [start angle=\capendangle, end angle=180-\capendangle, radius=\lenssphereradius];
    % "Equator"
    \tdplotCsDrawLatCircle[black, ultra thick]{\lenssphereradius}{90-\lenscapangle}

    % Axis
    \draw [->] (0, 0, \lenssphereradius) -- ++(0, 0, 0.5*\axislen) coordinate(Rot) -- ++(0, 0, 0.5*\axislen);
    % Rotation label
    \begin{scope}[shift=(Rot)]
      % Thicker version to break the axis, if desired.
      \draw [white, line width=2pt] (-240:0.7) arc [start angle=-240, end angle=60, radius=0.7];
      % Actual rotation with label
      \draw [->] (-240:0.7) arc [start angle=-240, end angle=60, radius=0.7] node [above right] {$2\pi q/p$};
    \end{scope}
  \end{scope}
  
  % Bottom cap
  \begin{scope}[shift=(SouthSphereOrigin)]
    \draw[tdplot_screen_coords, black] (-\capendangle:\lenssphereradius) arc [start angle=-\capendangle, end angle=-180+\capendangle, radius=\lenssphereradius];
  \end{scope}

 \end{tikzpicture}

\end{document}
